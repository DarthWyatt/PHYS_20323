%%%%%%%%%%%%%%%%%%%%%%%%%%%%%%%%%%%%%%%%%%%%%%%%%%%%%%%%%%%%
%%%%%%%%%%%%%%%%%%%%%%%%%%%%%%%%%%%%%%%%%%%%%%%%%%%%%%%%%%%%
%%%%%%%%%%%%%%%%%%%%%%%%%%%%%%%%%%%%%%%%%%%%%%%%%%%%%%%%%%%%
%%%%%%%%%%%%%%%%%%%%%%%%%%%%%%%%%%%%%%%%%%%%%%%%%%%%%%%%%%%%
%%%%%%%%%%%%%%%%%%%%%%%%%%%%%%%%%%%%%%%%%%%%%%%%%%%%%%%%%%%%
\documentclass[12pt]{article}
\usepackage{fancyhdr}
\usepackage{pslatex}
\usepackage{epsfig}
\usepackage{times}
\usepackage{amsmath}
\usepackage{mathrsfs}
\usepackage[dvipsnames]{xcolor}
\usepackage[hidelinks]{hyperref}%renewcommand{\topfraction}{1.0}
\renewcommand{\topfraction}{1.0}
\renewcommand{\bottomfraction}{1.0}
\renewcommand{\textfraction}{0.0}
\setlength {\textwidth}{6.6in}
\hoffset=-1.0in
\oddsidemargin=1.00in
\marginparsep=0.0in
\marginparwidth=0.0in                                                                               
\setlength {\textheight}{9.0in}
\voffset=-1.00in
\topmargin=1.0in
\headheight=0.0in
\headsep=0.00in
\footskip=0.50in                                         
\setcounter{page}{1}
\begin{document}
\def\pos{\medskip\quad}
\def\subpos{\smallskip \qquad}
\newfont{\nice}{cmr12 scaled 1250}
\newfont{\name}{cmr12 scaled 1080}
\newfont{\swell}{cmbx12 scaled 800}
%%%%%%%%%%%%%%%%%%%%%%%%%%%%%%%%%%%%%%%%%%%%%%%%%%%%%%%%%%%%
%     DO NOT CHANGE ANYTHING ABOVE THIS LINE
%%%%%%%%%%%%%%%%%%%%%%%%%%%%%%%%%%%%%%%%%%%%%%%%%%%%%%%%%%%%
%     DO NOT CHANGE ANYTHING ABOVE THIS LINE
%%%%%%%%%%%%%%%%%%%%%%%%%%%%%%%%%%%%%%%%%%%%%%%%%%%%%%%%%%%%
%     DO NOT CHANGE ANYTHING ABOVE THIS LINE
%%%%%%%%%%%%%%%%%%%%%%%%%%%%%%%%%%%%%%%%%%%%%%%%%%%%%%%%%%%%


\begin{center}
{\large
\bf PHYSICS  X0323: Fall 2025 - LaTeX Example
}\\
\end{center}

\noindent {1. At time t = 0 a particle is represented by the wave function} 

%\begin{center}

\begin{equation}
  \Psi(x) = \begin{cases}
    A\binom{x}{a}, & 0 \le x \le a \\[1em]
    A\binom{b-x}{b-a}, & a \le x \le b\\[1em]
    0, & \text{otherwise}
    \end{cases}
\end{equation}


%\end{center}

\vskip0.1in
\indent
where A, \textit{a} and \textit{b} are constants.

\vskip0.2in
\indent
(a) (3.3 points) Normalize $\Psi$ (i.e., find \textit{A} terms of \textit{a} and \textit{b}).

\vskip0.1in
\indent
(b) (3.3 points) Where is the particle likely to be found at \textit{t} = 0?.

\vskip0.1in
\indent
(c) (3.4 points) What is the expectation value of \textit{x}?.
%%%%%%%%%%%%%%%%%%%%%%%%%%%%%%%%%%%%%%%%%%%%%%%%%%%%%%%%%%%%%%%%%%
\vskip0.2in
\noindent
\textbf{2. The following questions refer to stars in the Table below.}
\vskip0.01in
\indent
\textit{Note: There may be multiple answers.}

\begin{center}
\begin{tabular}{|l|c|r|r|r|r|r|r|}\hline
Name & Mass & Luminosity & Lifetime & Temperature & Radius & Variable? \\\hline
$\delta$ Scu. & 2.0 $M_{\odot}$ &  & $5.0 \times 10^{8}$ years &  & 2.0 $R_{\odot}$ & Y \\\hline
$\gamma$ Del. & 0.7 $M_{\odot}$ &  & $4.5 \times 10^{10}$ years & 5000 K &  & N  \\\hline
$\beta$ Cyg. & 1.3 $M_{\odot}$ &  3.5 $L_{\odot}$ &  &  &  & Y  \\\hline
$\eta$ Car. & 60. $M_{\odot}$ & 10$^{6}$ $L_{\odot}$ & $8.0 \times 10^{5}$ years &  &  & Y \\\hline
$\epsilon$ Eri. & 6.0 $M_{\odot}$ & 10$^{3}$ $L_{\odot}$ &  & 20,000 K &  & N  \\\hline
$\alpha$ Cen. & 1.0 $M_{\odot}$ &  &  & 6000 K & 1.0 $R_{\odot}$ & N  \\\hline
\end{tabular}\vskip 0.2in
\end{center}

(a) (4 points) Which of these stars will produce a planetary nebula
\vskip0.1in
(b) (4 points) Elements heavier than Carbon will be produced in which stars.
%%%%%%%%%%%%%%%%%%%%%%%%%%%%%%%%%%%%%%%%%%%%%%%%%%%%%%%%%%%%%%%%%%
\vskip0.2in
\noindent
3. An electron is found to be in the spin state(in the z-basis): $\chi$ = \textit{A}$\binom{3i}{4}$

\vskip0.3in
\indent
(a) (5 points) Determine the possible values of \textit{A} such that the state is normalized.
\vskip0.1in
\indent
(b) (5 points) Find the expectation values of the operators \textcolor{red}{S$_{x}$}, \textcolor{purple}{S$_{y}$}, \textcolor{orange}{S$_{z}$} and S$^{2}$.

\vskip0.2in
\indent
The matrix representations in the z-basis for the components of electron spin operators are given by:

\vskip0.2in
\indent
\textcolor{red}{S$_{x}$=$\frac{h}{2}$$\binom{0 \; 1}{1 \; 0}$;}
\indent
\indent
\indent
\textcolor{purple}{S$_{x}$=$\frac{h}{2}$$\binom{0 \; -i}{i \; 0}$;}
\indent
\indent
\indent
\textcolor{orange}{S$_{x}$=$\frac{h}{2}$$\binom{1 \; 0}{0 \; -1}$;}

\vskip0.5in
\noindent
Latex Example

\end{document}